% !TEX TS-program = xelatex
\documentclass[12pt,a4paper]{report}

%---------------------------------------------------------------------------
% Packages
%---------------------------------------------------------------------------
\usepackage[margin=2.5cm]{geometry}
\usepackage{setspace}
\usepackage{lmodern}
\usepackage[T1]{fontenc}
\usepackage[utf8]{inputenc}
\usepackage{graphicx}
\usepackage{subcaption}
\usepackage{booktabs}
\usepackage{longtable}
\usepackage{float}
\usepackage{caption}
\usepackage{amsmath,amssymb}
\usepackage{hyperref}
\usepackage[numbers,sort&compress]{natbib}
\usepackage{tocloft}
\usepackage{fancyhdr}
\usepackage{listings}
\usepackage{xcolor}

%---------------------------------------------------------------------------
% Graphics and Listings Setup
%---------------------------------------------------------------------------
\graphicspath{{figures/}}
\captionsetup{font=small,labelfont=bf}
\lstset{
  basicstyle=\ttfamily\small,
  breaklines=true,
  frame=single,
  numbers=left,
  numberstyle=\tiny,
  keywordstyle=\color{blue},
  commentstyle=\color{gray},
  stringstyle=\color{teal},
  showstringspaces=false
}

%---------------------------------------------------------------------------
% Header and Footer
%---------------------------------------------------------------------------
\pagestyle{fancy}
\fancyhf{}
\fancyhead[LE,RO]{\thepage}
\fancyhead[RE]{\nouppercase{\leftmark}}
\fancyhead[LO]{\nouppercase{\rightmark}}
\renewcommand{\headrulewidth}{0.4pt}

%---------------------------------------------------------------------------
% Spacing
%---------------------------------------------------------------------------
\onehalfspacing

%---------------------------------------------------------------------------
% Document
%---------------------------------------------------------------------------
\begin{document}

%-------------------- Front Matter --------------------
\begin{titlepage}
	\centering
	{\Huge\bfseries Title of the Thesis\\[1em]}
	{\Large Subtitle if needed\\[4em]}
	{\Large Author Name\\[2em]}
	{\large Department, University\\[1em]}
	{\large Month Year\\[4em]}
\end{titlepage}

% Abstract
\begin{abstract}
	% Add Abstract to ToC
	\addcontentsline{toc}{chapter}{Abstract}
	% Your abstract text here.
	In professional cycling, athletes are often sent to training camps in locations such as the French Alps, Andorra, Sierra Nevada,
	Colorado Springs, or other high altitude destinations. While a big part of this is altitude acclimatisation, many cyclists report
	that the biggest effect is actually the suitability of the roads for training. Long, car-devoid mountains, where athletes can just
	put their head down and focus on their effort means that their training quality is improved, as opposed to having to ride through small
	villages, stopping at intersections and being constantly vigilant of overtaking cars or traffic furniture. This thesis explores the
	possibility of being able to track and quantize "Training Suitability" of roads. What is proposed is an integrated framework that
	can semantically partition continuous trajectory data from a cyclist's GPS device, and meaningfully define each segment in the context of
	training suitability. Starting from raw GPS traces, we apply geometric filtering and map-matching to correct noise and align positions with a
	reference network. An adaptive segmentation algorithm then identifies breakpoints using curvature statistics, speed variance,
	and road metadata (such as road name, or speed limit), yielding segments that have a homogenous quality.
	Each segment is characterized by spatial, temporal, and contextual features and subsequently classified through a Wavelet Transform
	function that quantizes variability and changepoints in data streams. The scalability and robustness of the framework will be evaluated
	through a small-scale live web application.

\end{abstract}

% Table of Contents, List of Figures, List of Tables
\tableofcontents
\listoffigures
\listoftables

%-------------------- Main Chapters --------------------

\chapter{Introduction}
\label{chap:introduction}
\section{Background}
% Background content here.

\section{Research Objectives}
% Objectives here.

\chapter{Objects of Relevance}
\label{chap:objects}
% Describe objects relevant to the project.

\chapter{Literature Review}
\label{chap:litreview}
\section{LSEPI Analysis}
% LSEPI analysis content.

% Sections for each component
\section{Component A}
% ...

\section{Component B}
% ...

% Add more as needed

\chapter{Requirements Document}
\label{chap:requirements}
\section{Software Requirements Specification (SRS)}
\label{sec:srs}
% ISO/IEC/IEEE 29148-compliant structure
\subsection{Purpose and Scope}
% Define the purpose of this SRS and the scope of the system.

\subsection{Intended Audience and Reading Suggestions}
% Stakeholders and recommendations for reading order.

\subsection{Overall Description}
\subsubsection{Product Perspective}
% Context and interfaces with other systems.

\subsubsection{Product Functions}
% Summary of major functions of the system.

\subsubsection{User Characteristics}
% User profiles, skill levels, and constraints.

\subsubsection{Operating Environment}
% Hardware, software, and regulatory environments.

\subsubsection{Design and Implementation Constraints}
% Standards, languages, and tools that constrain design.

\subsubsection{Assumptions and Dependencies}
% External factors assumed to be true.

\subsection{Specific Requirements}
\subsubsection{External Interface Requirements}
% \paragraph{User Interfaces} Description of UI requirements.
% \paragraph{Hardware Interfaces} Hardware communication needs.
% \paragraph{Software Interfaces} APIs and protocols.
% \paragraph{Communication Interfaces} Network protocols and formats.

\subsubsection{Functional Requirements}
% Enumerate functional requirements (ID, description, rationale).

\subsubsection{Performance Requirements}
% Timing, throughput, and capacity requirements.

\subsubsection{Logical Database Requirements}
% Data definitions, schemas, and constraints.

\subsubsection{Software System Attributes}
\paragraph{Reliability}
% Availability, MTBF, and error handling.

\paragraph{Availability}
% Uptime requirements.

\paragraph{Security}
% Authentication, authorization, and encryption.

\paragraph{Maintainability}
% Modularity and supportability requirements.

\paragraph{Portability}
% Platform portability requirements.

\subsubsection{Other Requirements}
% Safety, legal, and standards compliance.

\section{Standards and Compliance}
% List standards adhered to.

\chapter{Methodology}
\label{chap:methodology}
% Chronological methodology

\section{Design}
\label{sec:design}
% ISO/IEC/IEEE 42010 compliant design documentation

\subsection{Scope and Purpose}
Define the objectives and deliverables of the design phase, linking back to the SRS.

\subsection{Architectural Design}
\subsubsection{Reference Architecture}
Present a high-level system architecture diagram using UML or SysML.

\paragraph{Architectural Views}
\begin{itemize}
	\item \textbf{Context View}: System boundaries, external entities, and interfaces.
	\item \textbf{Functional View}: Key modules and their responsibilities.
	\item \textbf{Physical View}: Hardware deployment and network topology.
	\item \textbf{Information View}: Data flow and storage structures.
	\item \textbf{Behavioral View}: Sequence and state diagrams for critical scenarios.
\end{itemize}

\subsubsection{Design Rationale}
Justify architectural decisions, discuss alternatives, and document trade-offs.

\subsubsection{Design Patterns and Styles}
Identify and describe any applied patterns (e.g., MVC, layered) and coding conventions.

\subsection{Module-Level Design}
\subsubsection{Module Decomposition}
List all system modules, their interfaces, and dependencies. Include a module dependency diagram.

\subsubsection{Module Specifications}
For each module:
\begin{description}
	\item[Name:] Purpose and description.
	\item[Interfaces:] Inputs, outputs, and protocols.
	\item[Behavior:] Algorithmic overview or pseudocode.
	\item[Dependencies:] Internal and external module links.
\end{description}

\subsection{Interface Design}
\subsubsection{User Interface}
Wireframes or mockups, navigation flows, and accessibility guidelines (e.g., WCAG 2.1).

\subsubsection{Software Interfaces}
API contracts, message schemas (e.g., JSON, XML), and error handling.

\subsubsection{Hardware Interfaces}
Detailed electrical and protocol specifications for device interactions.

\subsection{Data Design}
\subsubsection{Data Models}
ER diagrams or class diagrams showing data entities and relationships.

\subsubsection{Database Schema}
Detailed table definitions, keys, indexes, and normalization rules.

\subsubsection{Data Dictionary}
Definitions and formats for all data elements used in the system.

\subsection{Security and Safety Design}
Risk assessment, threat models, and mitigation strategies.

\subsection{Standards and Compliance in Design}
List all applicable ISO/IEC, IEEE, and domain-specific standards adhered to (e.g., ISO/IEC 27001 for security).

\section{Implementation}
% Implementation details.

\section{Testing and Results}
% Testing procedures and results.

\chapter{Results and Conclusions}
\label{chap:results}
% Final results and conclusions.

\chapter{Further Discussions and Research Gaps}
\label{chap:discussion}
% Discussion and gaps.

\appendix
\chapter{Appendix}
\label{chap:appendix}
\section{Further Reading}
% References for further reading.

\section{Source Code}
% Include code listings or reference external files.

\section{File Structure}
% Describe directory/file organization.

\section{Additional Documentation}
% Any extra docs.

%---------------------------------------------------------------------------
% Bibliography
%---------------------------------------------------------------------------
\bibliographystyle{unsrtnat}
\bibliography{references}

\end{document}

