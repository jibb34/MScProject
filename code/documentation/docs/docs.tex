\documentclass{article}
\usepackage[utf8]{inputenc}
\usepackage{hyperref}
\usepackage{listings}
\usepackage{xcolor}

% Define colors for code listings
\lstset{
  basicstyle=\ttfamily\small,
  keywordstyle=\color{blue},
  stringstyle=\color{purple},
  commentstyle=\color{gray},
  showstringspaces=false,
  breaklines=true,
  frame=single,
}

\title{Documentation: GPX-to-JSON Normalizer}
\author{Auto-generated}
\date{\today}

\begin{document}
\maketitle

\section{Overview}
This tool reads a GPX file and converts each track point into a compact JSON object. It keeps only the essential fields (latitude, longitude, elevation, time) and captures any extension tags (like power, heart rate, cadence) under the \texttt{<extensions>} element.

\section{Environment Setup}
\begin{itemize}
	\item Python 3.x
	\item Install dependencies:
	      \begin{lstlisting}
pip install gpxpy
    \end{lstlisting}
\end{itemize}

\section{Usage}
Run the script from the command line:
\begin{lstlisting}
python gpx_to_json.py <input.gpx> <output.json>
\end{lstlisting}

\noindent
Where:
\begin{itemize}
	\item \texttt{input.gpx} is the source GPX file.
	\item \texttt{output.json} is the JSON file to create.
\end{itemize}

\section{Code Reference}
Below is the full script with explanations.

\subsection{Imports and Setup}
\begin{lstlisting}[language=Python]
import argparse
import os
import json
import gpxpy
\end{lstlisting}
These modules provide:
\begin{itemize}
	\item \texttt{argparse} for parsing command-line options.
	\item \texttt{os} to create directories if needed.
	\item \texttt{json} to write the output file.
	\item \texttt{gpxpy} to read GPX files.
\end{itemize}

\subsection{Argument Parsing}
\begin{lstlisting}[language=Python]
def parse_args():
    parser = argparse.ArgumentParser(
        description="Normalize a GPX file to a compact JSON list of track-point objects"
    )
    parser.add_argument("gpx_file", help="Path to the GPX file to read")
    parser.add_argument("output_json", help="Path to write the normalized JSON output")
    return parser.parse_args()
\end{lstlisting}
This function sets up two required arguments: the input GPX file and the output JSON path.

\subsection{Loading GPX Data}
\begin{lstlisting}[language=Python]
def load_gpx(path):
    with open(path, 'r') as f:
        return gpxpy.parse(f)
\end{lstlisting}
Opens the given file path and uses \texttt{gpxpy} to parse it into a GPX object.


\subsection{Converting Points}
\begin{lstlisting}[language=Python]
def gpx_to_points(gpx):
    points = []
    for track in gpx.tracks:
        for segment in track.segments:
            for pt in segment.points:
                # Each point must have at least latitude and longitude
                point = {
                    "lat": pt.latitude,
                    "long": pt.longitude,
                    "elv": pt.elevation if pt.elevation is not None else None,
                    "time": pt.time.isoformat() if pt.time else None,
                }

                # include extensions
                if pt.extensions:
                    for ext in pt.extensions:
                        # If an extension does have children, parse them individually, otherwise add the key:value pair directly
                        if any(True for _ in ext):
                            for child in ext:
                                point[child.tag.split("}")[-1]] = child.text
                        else:
                            tag = ext.tag.split("}")[-1]
                            text = (
                                ext.text.strip()
                                if ext.text and ext.text.strip()
                                else None
                            )
                            if text:
                                point[tag] = text

                # Add parsed point to list
                points.append(point)

    return points
\end{lstlisting}
This function:
\begin{enumerate}
	\item Iterates each track, segment, and point.
	\item Builds a dict with rounded coordinates, elevation, and ISO time.
	\item Reads \texttt{<extensions>}: if there are nested tags, it parses each child; otherwise, it takes the direct text.
	\item Appends each point dict to a list.
\end{enumerate}

\subsection{Writing JSON Output}
\begin{lstlisting}[language=Python]
def write_json(points, path):
    os.makedirs(os.path.dirname(path) or '.', exist_ok=True)
    with open(path, 'w') as f:
        json.dump(points, f, separators=(',', ':'), ensure_ascii=False)
\end{lstlisting}
Creates any needed folders and writes a compact JSON with no extra spaces.

\subsection{Main Entry Point}
\begin{lstlisting}[language=Python]
def main():
    args = parse_args()
    gpx = load_gpx(args.gpx_file)
    points = gpx_to_points(gpx)
    write_json(points, args.output_json)

if __name__ == '__main__':
    main()
\end{lstlisting}
This ties everything together: parse arguments, load the GPX, convert points, and write the JSON.

\end{document}

