\documentclass{article}
\usepackage[utf8]{inputenc}
\usepackage{hyperref}
\usepackage{listings}
\usepackage{xcolor}

% Define colors for code listings
\lstset{
  basicstyle=\ttfamily\small,
  keywordstyle=\color{blue},
  stringstyle=\color{purple},
  commentstyle=\color{gray},
  showstringspaces=false,
  breaklines=true,
  frame=single,
}

\title{Documentation: GPX-to-JSON Normalizer}
\author{Auto-generated}
\date{\today}

\begin{document}
\maketitle

\section{Map Preprocessing and Graph Construction}

To enable candidate edge detection and Hidden Markov Model (HMM)-based path inference, the road network must be represented as a graph. This section describes how OpenStreetMap (OSM) data is either dynamically fetched or loaded from disk and filtered for bicycle-compatible roads.

\subsection{Input Format}
The program accepts a JSON file consisting of sequential GPS observations. Each observation is a dictionary with spatial and contextual metadata such as latitude, longitude, elevation, timestamp, temperature, heart rate, cadence, speed, and heading.

\begin{lstlisting}
{
  "lat": 38.821468,
  "long": -104.862572,
  "elv": 1886.6,
  "time": "2024-08-06T13:12:10+00:00",
  "atemp": "20",
  "hr": "104",
  "cad": "62",
  "speed": 0.0,
  "heading": 0.0
}
\end{lstlisting}

\subsection{Graph Loading}
The road network can be supplied in two ways:

\begin{enumerate}
	\item \textbf{From a GraphML File:} If the user provides a file using \texttt{--graphml}, the graph is loaded via \texttt{osmnx.load\_graphml()}.
	\item \textbf{From OSM Bounding Box:} If no file is provided, the code calculates the bounding box of all valid GPS coordinates and queries OSM via \texttt{osmnx.graph\_from\_bbox()}.
\end{enumerate}

\subsection{Road Type Filtering}
Only road types usable by cyclists are retained in the graph. This includes:
\begin{itemize}
	\item \texttt{cycleway}
	\item \texttt{primary, secondary, tertiary}
	\item \texttt{residential, unclassified, service}
\end{itemize}

Edges not matching these categories are removed to improve routing accuracy and reduce computational overhead.

\subsection{Caching}
To avoid repeated API calls during development, OSM data is cached locally in a \texttt{./cache} folder.

\subsection{Execution}
Run the script with either:

\begin{lstlisting}
# Load from GraphML file
python get_map_data.py test.json --graphml my_graph.graphml

# Download from OSM based on GPS bounding box
python get_map_data.py test.json
\end{lstlisting}

\end{document}


