\documentclass{article}
\usepackage[margin=1in]{geometry}
\usepackage{hyperref}
\usepackage{listings}
\begin{document}
\title{Segmentation Engine Documentation}
\author{}
\date{}
\maketitle

\section{Overview}
The segmentation engine is a C++ HTTP service that performs route analysis and segment extraction. The server loads its configuration from \texttt{config/settings.json} and exposes a set of REST endpoints.

\section{Building and Running}
\begin{verbatim}
cmake -S . -B build
cmake --build build
./build/segmentation_engine  % binary name may vary
\end{verbatim}
The default server port and endpoint list are defined in \texttt{config/settings.json}.

\section{API}
\subsection{\texttt{POST /upload}}
Save a client-provided map JSON file.
\begin{description}
\item[Body] JSON object representing the map data.
\item[Response] \texttt{\{ "ok": true, "file": "uploads/map\_<timestamp>\_<rand>.json" \}}
\end{description}

\subsection{\texttt{GET /view?map=<path>}}
Render an interactive HTML viewer for a previously uploaded map.
\begin{description}
\item[Query] \texttt{map} -- path returned from \texttt{/upload} (must be under \texttt{uploads/}).
\item[Response] HTML page comparing raw OSRM geometry to derived route signal.
\end{description}

\subsection{\texttt{GET /viewLab?map=<path>}}
Redirect to the static Signal Lab UI.
\begin{description}
\item[Query] optional \texttt{map} path.
\item[Response] HTTP 302 redirect to \texttt{/static/signal\_lab.html}.
\end{description}

\subsection{\texttt{GET /lab/meta?map=<path>}}
List available uploads and optionally preload one map.
\begin{description}
\item[Query] optional \texttt{map} path.
\item[Response] JSON \texttt{\{ "uploads": ["uploads/..."], "preload": "uploads/..."? \}}.
\end{description}

\subsection{\texttt{POST /lab/resample}}
Build a route signal and resample requested variables onto a uniform distance grid.
\begin{description}
\item[Body] \texttt{\{ "map": "uploads/...", "ds\_m": <step\_m>, "vars": ["elev", ...] \}}
\item[Response] \texttt{\{ "ok": true, "s\_km": [..], "ds\_m": <step\_m>, "series": {"var": [...]}\}}
\end{description}

\subsection{\texttt{POST /wavelet}}
Run wavelet-based terrain segmentation on a map and optionally persist results.
\begin{description}
\item[Body] \texttt{\{ "map": "uploads/...", "fn": "terrain", "persist": false \}}
\item[Response] JSON containing \texttt{segments} with segment geometry and metadata, \texttt{series} with processed signals, \texttt{s\_km\_uniform} axis and \texttt{ds\_m} spacing. If database persistence fails, a \texttt{db\_error} field is included.
\end{description}

\subsection{\texttt{GET /segments}}
Fetch previously persisted segments intersecting a bounding box.
\begin{description}
\item[Query] either \texttt{bbox=west,south,east,north} or \texttt{minLon}, \texttt{minLat}, \texttt{maxLon}, \texttt{maxLat}.
\item[Response] GeoJSON \texttt{FeatureCollection} of matching segments.
\end{description}

\end{document}
